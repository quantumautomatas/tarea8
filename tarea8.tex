\documentclass{article}

% formato
\usepackage[margin = 1.5cm, letterpaper]{geometry}
\usepackage[utf8]{inputenc}

%tablas
\usepackage{graphicx}

%formato ecuaciones
\usepackage{amsmath}

% símbolos
\usepackage{amssymb}

% manejo de tablas
\usepackage{float}

\begin{document}
    \title{
        Autómatas y Lenguajes formales \\
        Ejercicio Semanal 8
    }

    \author{
        Sandra del Mar Soto Corderi \\
        Edgar Quiroz Castañeda
    }

    \date{
        5 de abril del 2019
    }
    
    \maketitle

    \begin{enumerate}
        \item {
        Dado el siguiente lenguaje: $L = \{w \in \{a,b\}^* | \eta_a(w) = \eta_b(w) \}$ es decir el lenguaje que tiene el mismo número de a's que de b's
        \begin{enumerate}
        	\item {
        	Demuestra que el lenguaje no es regular con el lema del bombeo.\\
        	
        	El lema del Bombeo dice que si L es un lenguaje regular infinito entonces existe un número n $\in \mathbb{N}$, llamado constante de bombeo para L, tal que para cualquier cadena w $\in L$ con $|w| \geq$ n existen cadenas u,v,x tales que:        	
        	\begin{enumerate}
        		\item {
        		w = uvx
        		}
        		\item {
        		$|uv| \geq$ n
        		}
       		 	\item {
        		v $\neq \epsilon$
        		}
    			\item {
    			$\forall m \in \mathbb{N} (uv^mx \in L)$
    			}
        	\end{enumerate}
        	
        	Vamos a demostrar por contradicción:\\
        	
        	Supongamos que L es un lenguaje regular y que n es la constante de bombeo tal que cualquier cadena $w \in L$, $|w| \geq n$. Tomemos una w que claramente está en L: $w = a^nb^n$, y demos la siguiente descomposición de w en uvx:
        	
        	$u = a^k, \ v= a^j, \  k \geq 0, j \geq 1 \ $ 
        	De ahí, tenemos que $x = a^{n-k-j}b^n$\\
        	
        	Se puede ver que cumplimos con  $|uv| \geq$ n y v $\neq \epsilon$. Si tomamos $m = 4$, por el lema del bombeo, se debe cumplir que $uv^4x \in L$.\\
        	
        	Pero tenemos que $uv^4x = a^ka^ja^ja^ja^ja^{n-k-j}b^n = a^{n+3j}b^n \notin L$, lo cual es una contradicción.\\
        	
        	Por lo tanto, L no es un lenguaje regular $\blacksquare$
        	}\\
        
        	\item{
        	Demuestra que el lenguaje no es regular usando el conjunto estafador.\\
        	
        	Un conjunto infinito $S \subseteq \Sigma^*$ es un conjunto estafador para L si y sólo si $\forall x,y \in S (x \not\equiv_L y)$.\\
        	
        	Sea $S = \{ a^i | i \in \mathbb{N}\}$, veamos que S es un conjunto estafador: Sean $a^m , a^n \in S \ \text{con} \ n \neq m$\\
        	
        	Por un lado tenemos $ a^mb^m \in L$\\
        	Por otra parte tenemos $a^nb^m \not \in L$\\
        	
        	Por lo tanto $a^m \not\equiv_L a^n$ y S es un conjunto estafador de L.\\
        	
        	Como pudimos encontrar un conjunto estafador de L, concluimos que L no es regular. $\blacksquare$
        	}
        \end{enumerate}
    	}
    \end{enumerate}
\end{document}